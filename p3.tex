\documentclass{article}
\usepackage{graphicx} % Required for inserting images
\usepackage{algorithm}
\usepackage{algpseudocode}
% \usepackage{amsthm}

\title{Bellman-Ford Algorithm}
\author{Aaron Frye}
\date{November 2024}

\begin{document}

\maketitle

\section{Overview}
The Bellman-Ford algorithm is a method of computing the shortest path from a given source node to each other node in a graph. It is similar to Dijkstra's algorithm in it's function, but has the advantage of being able to handle graphs with negative edge weights. The algorithm can fail to give a correct answer when the graph has negative cycles, as when these are present there will be nodes with no finite shortest path from $s$.

\section{Pseudocode}

\begin{algorithm}
    \caption{Bellman-Ford algorithm}
    \textbf{Inputs}: A weighted digraph with negative weights $G(V, E)$, and a start node $s$ from $V$ to compute shortest paths from. \\
    \textbf{Outputs}: The minimum cost to reach each node from $s$
    \begin{algorithmic}
    \For{$n \in G$}
        \If{$n == s$}
            \State $n.cost \gets 0$
        \Else
            \State $n.cost \gets \infty$
        \EndIf
    \EndFor

    \For{$i \gets 0\ to\ |V|-1$}
        \For{$e(u,v) \in E$}
            \If{$v.cost < u.cost + e.weight$}
                \State $v.cost \gets u.cost + e.weight$
            \EndIf
        \EndFor
    \EndFor
    \end{algorithmic}
\end{algorithm}


\section{Data Structures}
Our implementation uses a networkx graph as it's only data structure, but many of it's functionality could be easily represented with other more fundamental structures. \\ 

Computing the cost of reaching each node could be implemented as a hash map, which would allow constant time lookup and updating. Similarly, we could store the edges as $(source,\ destination)$ pairs in a simple array, and use a hash map to store edge weights.


\section{Intuition of Correctness}

It can be shown that for a weighted digraph with negative weights but no negative cycles, and for a given node $n$, $n$ will have at least one outgoing edge $e$ which will be the optimal path from $n$ to the destination node of $e$. Using this fact, we can prove by induction that with each iteration of our algorithm we will find the optimal cost to reach at least one new node from $s$, \\

Base Case: Using the above fact, we will find that there is at least one outgoing edge $e$ connecting $s$ to some other node $a$ for which $e$ is the optimal path from $s$ to $a$. Clearly $a$ will have it's cost updated to the optimal value.

Inductive Case: Assume that after $k$ iterations we find the optimal cost for some node set of nodes $N = \{n_1,\ n_2,\ ...\ n_m\}$. For each node in this set we know there will be an outgoing edge $e_i$ from $n_i$ to some other node $n_i'$ for which $e_i$ is on the optimal path from $n_i$ to $n'_i$. For any of these $n'_i$, it's optimal path must include an edge from the $e_i$, otherwise our graph would have to include a negative cycle for similar reasoning to the above lemma.

\section{Time Complexity}
Our algorithm consists of two for loops. The outer loop iterates $V-1$ times, while the inner loop runs $E$ times. Since in the worst case $E$ can be $V^2$, our algorithm should run in $O(V^3)$ time.

\section{Space Complexity}
Our algorithm needs to keep track of the cost of reaching each node and the weight, destination, and source of each of it's edges. If these are implemented in a hash map as described above, there should only need to be $V$ entries in the cost hash map. There will be $E$ entries in the edge weight hash map, plus the array of $(source,\ destination)$, so our overall space complexity will be $O(V + E)$. Since at worst $E = V^2$ our space complexity is better approximated by $O(V^2)$.

\section{Code}

\section{Results}

\section{Questions for Dr. Javidian}
\begin{itemize}
    \item What should the code output? The actual path or is the costs enough?
    \item How rigorous should my correctness be?
    \item Can we just use a networkx graph for our data structure?
\end{itemize}

\section{TODO}
\begin{itemize}
    \item Add $\Theta$ case for space complexity, specify number of nodes and number of edges, also do this for space complexity
    \item Add visuals and explain test cases
    \item Rework intuition of correctness
\end{itemize}

\end{document}
